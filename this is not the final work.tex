\documentclass[a4paper,12pt]{article}
\begin{document}
\title{AN ENVIRONMENT FOR BOARD GAME COMPETITIONS IN SCALA SUITABLE FOR PARTS B AND C}
\date{\today}

\begin{tabular}{|l|l|}
\hline
Name & RegNo: \\
\hline
Mutonerwa Stephen Musaazi & 15/U/744 \\
\hline
 KISEMBO RITAH & 14/U/8028/PS \\
\hline
 KAVUMA TIMOTHY	& 15/U/6100/PS \\ 
\hline
 TAREMWA JOAB	& 15/U/13099/EVE \\
\hline
\end{tabular}


\maketitle
\pagenumbering{arabic}

\newpage
\section{Introduction}
\section{  Background to the problem}

Creating interfaces for board Games is the most disturbing task in game development, and many software developers find it difficult to draw boards using the existing Board Game Designing  studio software for example CardWaden. 
Programmers use IDE (Integrated Development Environment) such as Python, Java, to develop codes that control, game interfaces
But if we come up with an environment comprising of board game design, reusable interface, and the software itself that software developers can edit according to what q1they want their games to behave.

\section{ Problem Statement}
The project will tackle is to set up an environment that allows for machine verses machine competitions in the board game.  
In addition to that, programmers waste much time developing the code that can control the interface. Programmers sit and make prototypes of the board games, they also test different output of these prototype at all stages of the game.
But just like developing a machine verses a machine competition board game, with a reusable interface.  The interface is to be made easy to use for game engines.
\section{	Objective}
The long-term objective of this project is to provide a reusable interface where future students can pit their game-playing engines against each other. The focus here is on good object-oriented design, reusability, make easy to use interfaces for game engines, and an appealing graphical user interface. Also create two (possibly simple) engines to test the software.


\subsection{Aim or general objective}
To analyze the user requirements of the new system and identify the problems game developers have been facing. 
To design a reusable board game interface. 
\section{Research Scope}
The target of this project is to people with some knowledge of programming. 
This people should be able to edit codes, and interfaces as well. 
They should also have some knowledge of Photoshop.
	
\section{Methodology}
\subsection{Research Methods}
This information was collected by interviewing computer science students, especially those specializing game development. Some students made reports that using CardWaden (software for designing interfaces) to design beautiful board designs or user interfaces is the major challenge they face. This is because that it needs some knowledge and skills in using CardWaden which many of them don’t have. Others said that designing programming source codes for board games is so challenging.  Also others said that pacing and players win is too challenging because this will change over the course of the game, but having a sense of what the objective of the game is, and how the players win/lose, is important, also condition how the game ends. Keep in mind how long you want the game to be and how the win conditions and game end condition tie into that. I mentioned this separate point because the pacing, end, and win conditions are intrinsic and core to the entire game experience.
\section{References}{
This is missing 
}

\end{document}
