\documentclass[a4paper,6pt]{article}
\begin{document}
\title{AN ENVIRONMENT FOR BOARD GAME COMPETITIONS IN SCALA SUITABLE FOR PARTS B AND C}
\date{\today}

\begin{tabular}{|l|l|}
\hline
Name & RegNo: \\
\hline
Mutonerwa Stephen Musaazi & 15/U/744 \\
\hline
 KISEMBO RHITAH & 14/U/8028/PS \\
\hline
 KAVUMA TIMOTHY	& 15/U/6100/PS \\
\hline
 TAREMWA JOAB	& 15/U/13099/EVE \\
\hline
\end{tabular}


\maketitle
\pagenumbering{arabic}

\section{Introduction}
This project will involve setting up an environment that allows machine-versus-machine competitions in the board game Hex.Creating interfaces for board Games is the most disturbing task in game development, and many software developers find it difficult to draw boards using the existing Board Game Designing  studio software for example CardWaden.
Programmers use IDE (Integrated Development Environment) such as Python, Java, to develop codes that control, game interfaces. But if we come up with an environment comprising of board game design, reusable interface, and the software itself that developers can edit basing on how they want their games to function and behave.
\section{  Background to the problem}
The game of Hex was first invented in 1942 by Piet Hein, a Danish scientist, mathematician, writer, and poet. In 1948, John Nash at Princeton re-discovered the game, which became popular among the math graduate students at Princeton. They called Hex either “Nash” or “John”, though the latter referred to the hexagonal bathroom tiles that they played the game on. In 1952, Parker Brothers, Inc. popularized the game as “Hex.The board game allows two players to play on a hexagonal grid, theoretically of any size and several possible shapes, though the typical size is 11 X 11. The players alternate turns with the goal of forming an unbroken chain of tiles of his own colour of course linking his two regions.
Currently according to hexagon.org,Hexagon has increasingly been interpreted for computer play whether online or simply on the desktop that strict adherence to board styles, colour schemes and number of players has fluctuated greatly. Two options are available when playing on the computer:one can either play with the computer program or connect to the internet and play with some one else where in the world.
\section{ Problem Statement}
The project will tackle how to set up an environment that allows for machine-verses-machine competitions in the board game.This will help programmers overcome time wastage on developing the code that controls the interfaces,since they sit and make prototypes of the board games, they also test different output of these prototypes at all stages of the game. Developing a machine verses a machine competition board game, with a reusable interface will ease their work .The interface is to be made easy to use for game engines.
\section{Objective}
The long-term objective of this project is to provide a reusable interface where future students can pit their game-playing engines against each other. The focus here is on good object-oriented design, reusability, make easy to use interfaces for game engines, and an appealing graphical user interface. Also create two (possibly simple) engines to test the software.
\subsection{Aim or general objective}
To analyze the user requirements of the new system and identify the problems game developers have been facing.
To design a reusable board game interface.
\section{Research Scope}
The target of this project is to people with some knowledge of programming.These people should be able to edit codes, and interfaces as well.They should also have some knowledge about graphics.
\section{Methodology}
\subsection{Research Methods}
This information was collected by interviewing computer science students, especially those specializing game development. Some students made reports that using CardWaden (software for designing interfaces) to design beautiful board designs or user interfaces is the major challenge they face. This is because that it needs some knowledge and skills in using CardWaden which many of them don’t have. Others said that designing programming source codes for board games is so challenging.  Also others said that pacing and player's win is too challenging because this will change over the course of the game, but having a sense of what the objective of the game is, and how the players win/lose, is important, also condition how the game ends. Keep in mind how long you want the game to be and how the win conditions and game end condition tie into that. I mentioned this separate point because the pacing, end, and win conditions are intrinsic and core to the entire game experience.
\section{References}{
\begin{thebibliography}{1}
\bibitem{}{university Of Oxford.(2017). undergraduate studentproject[online].Available:https://www.cs.ox.ac.uk/teaching/studentprojects/516.html}
\bibitem{}{Play Hexgon[online].Available:http://www.hexagongame.org en.wikipedia.org/wiki/Hex(boardgame)}
\end{thebibliography}
}
\end{document}
